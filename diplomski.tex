\documentclass[times, utf8, diplomski, numeric]{fer}
\usepackage{booktabs}
\usepackage{graphicx}
\graphicspath{ {figures/} }

\begin{document}

% TODO: Navedite broj rada.
\thesisnumber{1467}

% TODO: Navedite naslov rada.
\title{Rekonstrukcija mikroskopskog objekta metodom sažimajućeg očitavanja}

% TODO: Navedite vaše ime i prezime.
\author{Ana Škaro}

\maketitle

% Ispis stranice s napomenom o umetanju izvornika rada. Uklonite naredbu \izvornik ako želite izbaciti tu stranicu.
\izvornik

% Dodavanje zahvale ili prazne stranice. Ako ne želite dodati zahvalu, naredbu ostavite radi prazne stranice.
\zahvala{Hvala.}

\tableofcontents

\chapter{Uvod}

U laserskoj mikroskopiji promatrani objekt se osvjetljava koherentnim laserskim svjetlom određene valne duljine, a optički sustav mikroskopa zasnovan je na Fresnelovoj optici. To znači da se umjesto klasičnih leća koriste rupičaste odnosno rešetkaste strukture. Ukoliko je promatrani objekt proziran, s različitom brzinom prostiranja svjetlosti od vakuuma, moguće je rekonstruirati sliku objekta koristeći informaciju o kutu za koji objekt zakreće svjetlost lasera. Navedeni postupak opisan je u radu \citep{Morgan:11}. Postavljena je pretpostavka da je pomoću navedenog postupka i sažimajućeg očitavanja, moguće razviti simulacijski model koji bi bio u stanju rekonstruirati sliku visoke rezolucije.

\chapter{Optički sustav mikroskopa}



\section{Fresnelova optika}

Modeliranje optičkog sustava mikroskopa započinje modeliranjem kretanja svjetlosti kroz prazni prostor.
Huygens - Fresnelovo načelo nazvano prema nizozemskom fizičaru Huygensu i francuskom fizičaru Fresnelu govori kako se svaka točka valne fronte može smatrati izvorom novog kuglastog vala (Slika \ref{Huygens-Fresnel-principle}). Prethodno načelo objašnjava nastajanje interferencijskih pruga prilikom difrakcije (ogiba) svjetlosti na uskoj pukotini. !!prugeslika

\begin{figure}[h]
	\includegraphics[scale=0.8]{Huygens-Fresnel-principle.png}
	\centering
	\caption{Slika prikazuje Huygens-Fresnelovo načelo.}
	\label{Huygens-Fresnel-principle}
\end{figure}

Promatraju se dvije ravnine $(\xi,\eta)$ označena kao $P_0$ i $(x,y)$ označena kao $P_1$. Definirno je da svjetlost prolazi sustavom u pozitivnom smjeru $z$-osi, koja probada obje ravnine u njihovom ishodištu. Cilj je izračunati valno polje na ravnini $P_1$. 

Pretpostavimo li da je udaljenost izvorišne točke svjetlosti (na ravnini $P_0$)do točke upada svjetlosti na senzor ($P_1$ ravnina) mnogo veća od valne duljine svjetlosti $r_{01} \gg \lambda$, Huygens-Fresnelovo načelo može se prikazati jednadžbom

\begin{equation}
	\label{Fresnel-1}
	U(P_0) = \frac{1}{j\lambda}\int\int U(P_1)\frac{exp(jkr_{01})}{r_{01}}\cos(\theta)ds,
\end{equation}

gdje je $\theta$ kut između normale $\vec{n}$ na ravninu $P_0$ i vektora $\vec{r_{01}}$ koji prati zraku svjetlosti između ravnine $P_0$ i $P_1$. 
Član $\cos(\theta)$ je dan izrazom

\begin{equation}
	\label{cos_theta}
	\cos(\theta) = \frac{z}{r_{01}}
\end{equation}

i jednadžba \ref{Fresnel-1} se može napisati kao

\begin{equation}
\label{Fresnel-2}
U(x,y) = \frac{z}{j\lambda}\int\int U(\xi,\eta)\frac{exp(jkr_{01})}{r_{01}^2}d\xi d\eta,
\end{equation}

gdje je udaljenost $r_{01}$ preko Pitagorinog poučka moguće odrediti s

\begin{equation}
\label{r01}
r_{01} = \sqrt{z^2 + (x - \xi)^2 + (y - \eta)^2}.
\end{equation}

Prilikom modeliranja ponašanja koherentne svjetlosti unutar mikroskopa moguće je iskoristiti nekoliko daljnjih aproksimacija i pojednostavljenja. 

Dijeljenjem izraza \ref{r01} sa $z$ dobije se

\begin{equation}
\label{r01_divided}
r_{01} = z\sqrt{1 + \left(\frac{x - \xi}{z}\right)^2 + \left(\frac{y - \eta}{z}\right)^2}.
\end{equation}

Udaljenost $r_{01}$ se može pojednostavniti aproksimacijom pomoću razvoja drugog korijena u red. Ako je broj $b$ manji od 1, razvoj drugog korijena dan je izrazom

\begin{equation}
\label{binomial_expansion}
\sqrt{1+b} = 1 + \frac{1}{2}b - \frac{1}{8}b^2 + ...
\end{equation}

Broj članova koji je potrebno uzeti da bi razvojem dobili zadovoljavajuću preciznost ovisi o veličini b. U našem slučaju dovoljno je uzeti samo prva dva člana razvoja u red:

\begin{equation}
\label{approximation_r01}
r_{01} \approx z\left[1+\frac{1}{2}\left(\frac{x-\xi}{z}\right)^2 + \frac{1}{2}\left(\frac{y-\eta}{z}\right)^2\right]\ldotp
\end{equation}

U izrazu \ref{Fresnel-2} se član $r_{01}$ pojavljuje na dva mjesta. Za kvadrirani član u nazivniku možemo zanemariti čak i drugi član razvoja u red i ostaviti samo prvi $z$. Međutim, kod drugog pojavljivanja $r_{01}$ (u eksponentu), njegova vrijednost se množi s valnim brojem $k$\footnote{$k=\frac{2\pi}{\lambda}$}, a kako čak i male promjene u fazi mogu promijeniti značajno vrijednost eksponencijale, uzimaju se oba člana razvoja u red. 
Kao rezultat se dobiva izraz

\begin{equation}
\label{Fresnel-final}
U(x,y) = \frac{e^{jkz}}{j\lambda z}\iint_{-\infty}^{m} U(\xi,\eta)\exp\left\lbrace j\frac{k}{2z}[(x-\xi)^2 + (y - \eta)^2]\right\rbrace d\xi d\eta \ldotp
\end{equation}

U izrazu \ref{Fresnel-final} su dodane i granice integriranja, odnosno definiran je konačan otvor pukotine.
Kada bi se dobiveni izraz promatrao u diskretnoj domeni, kao na računalu, sa sumama umjesto integrala, primjećuje se da je on računalno vrlo složen. Potrebno je za svaku točku na poziciji $(x,y)$ proći svaku točku na pozicijama određenim s $(\xi,\eta)$. Razmišljajući o tome, primjećuje se da izraz \ref{Fresnel-final} ustvari predstavlja konvoluciju, koju je moguće izraziti u obliku


\begin{equation}
\label{Fresnel-conv}
U(x,y) = \iint_{-\infty}^{\infty}U(\xi,\eta)h(x - \xi, y - \eta)d\xi d\eta,
\end{equation}

gdje je konvolucijska jezgra

\begin{equation}
\label{Fresnel-conv-kernel}
h(x,y) = \frac{e^{jkz}}{j\lambda z}\exp \left[\frac{jk}{2z}(x^2 + y^2)\right] \cdotp
\end{equation}

Na ovaj način se izračun valnog polja svjetlosti u ravnini $P_1$ svodi na množenje s konvolucijskim kernelom u frekvencijskoj domeni, što je korištenjem modernih alata (poput MATLAB-a) s ugrađenim funkcijama koje brzo provode Fourierovu transformaciju, moguće izvesti znatno brže nego običnim sumiranjem (integriranjem).


\section{Model tanke leće}

Leća je izrađena od materijala čiji je indeks refrakcije tj. indeks loma svjetlosti veći od indeksa loma u zraku i obično iznosi 1.5. To znači da je u tom sredstvu brzina širenja svjetlosti manja nego u zraku. Leću smatramo \textbf{tankom} ukoliko zraka koja u leću uđe na koordinatama $(x,y)$ iz nje izađe na približno istim koordinatama na drugoj strani. Može se reći da tanka leća jednostavno uzrokuje kašnjenje valne fronte za iznos proporcionalan debljini leće u odgovarajućoj točki. 

Jedna tanka leća prikazana je na slici ??. S $D_{max}$ je označena maksimalna debljina leće, a $D(x,y)$ je funkcija ovisnosti debljine leće o $(x,y)$ koordinatama. Ukupni pomak u fazi svjetlosnog vala može se zapisati kao

\begin{equation}
\label{thin-lens-phase-shift}
\phi(x,y) = knD(x,y) + k\left[D_{max} - D(x,y)\right],
\end{equation}

gdje je $n$ indeks loma svjetlosti u materijalu od kojeg je leća izrađena, a $k$ je valni broj. Tada se prvi član jednadžbe \ref{thin-lens-phase-shift} $knD(x,y)$ može smatrati faznim pomakom uzrokovanim lećom, a drugi član $k\left[D_{max} - D(x,y)\right]$ je fazni pomak uzrokovan prolaskom svjelosti kroz preostali slobodni prostor.

Kompleksno polje $U_l'(x,y)$ na ravnini neposredno iza leće može se izraziti pomoću

\begin{equation}
	\label{lens-complex-field}
	U_l'(x,y) = t_l(x,y)U_l(x,y).
\end{equation}

Član $t_l(x,y)$ izraza \ref{lens-complex-field} je fazna transformacija oblika

\begin{equation}
	\label{T}
	t_l(x,y) = \exp\left[jkD_{max}\right]\exp\left[jk(n - 1)D(x,y)\right].
\end{equation}

Prilikom modeliranja objekta postavljenog između optičke rešetke i senzora (mikroskopski uzorak) potrebno je prema izrazu \ref{T} izračunati funkciju debljine objekta $D(x,y)$. Na ovaj način moguće je modelirati objekt proizvoljnog oblika.

U nastavku ovog poglavlja su izvedeni izrazi za model plano--konveksne leće i prizme. 

!!dokaz, pronadi u papirima


\section{Rekonstrukcija objekta}

Prema \citep{Morgan:11}, objekt promatran pod laserskim mikroskopom moguće je rekonstruirati iz faznih gradijenata koristeći samo jedno uzorkovanje (ekspozicijiu). Autori postavljaju uzorak (objekt) između senzora i optičke rešetke (slika \ref{Morgan:11-xray-experiment}).


\begin{figure}[h]
	\includegraphics[scale=0.8]{xray-experiment-setup.jpg}
	\centering
	\caption{Sustav korišten u eksperimentu iz rada \cite{Morgan:11}. X-zrake upadaju na optičku rešetku, a objekt postavljen između rešetke i senzora (detektora) ih zakreće za određeni kut, koji se određuje mjerenjima. Slika je preuzeta iz rada \cite{Morgan:11}.}
	\label{Morgan:11-xray-experiment}
\end{figure}


Uzorak nastao prolaskom snopa laserske svjetlosti preko optičke rešetke koristi se kao referenca za mjerenje pomaka odnosno gradijenta faze uzrokovanog prolaskom svjetlosti kroz uzorak. 
Pomak u fazi $\varphi$ za svaku točku se pronalazi pomoću pomaka $S$ uzorka u odnosu na referentni uzorak, razdvojen na horizontalni pomak $S_x$ i vertikalni pomak $S_y$. 
Korištenjem jednostavne trigonometrije kutevi $\theta_x$ i $\theta_y$ mogu se izraziti kao

\begin{equation}
	\label{thetas}
	\tan\theta_x = \frac{S_x}{z}  \text{ i } \tan\theta_y = \frac{S_y}{z}.
\end{equation} 

Usporedbom slika sa uzorkom i bez, određuje se magnituda pomaka $S$ za svaki piksel unutar slike. Na slici \ref{Morgan:11-xray-shift-measurement} je prikazano kako se vrši mjerenje iz rada \citep{Morgan:11}. 

\begin{figure}[h]
	\includegraphics[scale=0.8]{xray-shift-measurement.jpg}
	\centering
	\caption{Crvenom bojom je prikazan referentni prozor, dok je ispitni prozor plave boje. Pomaci $S_x$ i $S_y$ se računaju pomoću kros-korelacije. Slika je preuzeta iz rada \cite{Morgan:11}.}
	\label{Morgan:11-xray-shift-measurement}
\end{figure}

Mjerenje pomaka $S_x$ i $S_y$ za svaki piksel unutar slike optička rešetka--uzorak rezultira s dvije slike diferencijala faze, kao što je vidljivo na slici \ref{Morgan:11-xray-phase-gradients}. 

\begin{figure}[h]
	\includegraphics[scale=0.8]{x-ray-phase-gradients.jpg}
	\centering
	\caption{a) $S_x$ i b) $S_y$ diferencijali faze. Slika je preuzeta iz rada \cite{Morgan:11}.}
	\label{Morgan:11-xray-phase-gradients}
\end{figure}

Autori dalje tvrde da za elektromagnetski val valnog broja $k$, fazna dubina uzorka $\varphi(x,y)$ skreće upadne zrake za kuteve $\theta_x$ i $\theta_y$ prema izrazima

\begin{equation}
	\label{thetas-phasegrad}
	\theta_x = \frac{1}{k}\frac{\partial\phi}{\partial x} \text{ i } \theta_y = \frac{1}{k}\frac{\partial\phi}{\partial y}.
\end{equation}

Kombinirajući izraze \ref{thetas} i \ref{thetas-phasegrad} te pretpostavljajući homogenost materijala od kojeg je sastavljen objekt, vrijedi da je $\varphi = -k\delta T$, gdje je $\delta$ indeks loma svjetlosti, a T projecirana debljina objekta. Gradijenti debljine mogu se tada pisati kao

\begin{equation}
	\label{thickness-grad}
	\frac{\partial T}{\partial x} = \frac{1}{\delta}\tan^{-1}\left(\frac{S_x}{z}\right) \text{ and } \frac{\partial T}{\partial y} = \frac{1}{\delta}\tan^{-1}\left(\frac{S_y}{z}\right) 
\end{equation}

Jednadžbom \ref{thickness-grad} se dvije slike pomaka (\ref{Morgan:11-xray-phase-gradients}) pretvaraju u slike gradijenta debljine objekta. Integriranjem dobivenih slika gradijenata debljine se dobiva konačna debljina $T(x,y)$. Autori \cite{Morgan:11} pritom koriste Fourierovu transformaciju $\mathcal{F}$ i derivacijski teorem,

\begin{equation}
	\label{Fourier-derivative-theorem}
	\mathcal{F}\left(\frac{\partial T}{\partial x}\right) = ik_x\mathcal{F}(T).
\end{equation}

Fourierova transformacija definirana je izrazom

\begin{equation}
\label{Fourier-definition}
\mathcal{F}\left(T(x,y)\right) = \frac{1}{2\pi}\int_{-\infty}^{\infty}\int_{-\infty}^{\infty}T(x,y)e^{-i(k_xx+k_yy)}dxdy.
\end{equation}

Parametri $(k_x,k_y)$ su Fourierove koordinate odgovarajuće $(x,y)$. Projecirana debljina uzorka (objekta) se tada može izračunati korištenjem dviju slika gradijenata debljine ($\frac{\partial T}{\partial x}$ i $\frac{\partial T}{\partial y}$).

Autori rada \cite{Morgan:11} dolaze do konačnog izraza

\begin{equation}
	\label{Morgan-T-definition}
	T = \mathcal{F}^{-1}\left[\frac{\mathcal{F}\left(\frac{\partial T}{\partial x} + i\frac{\partial T}{\partial y}\right)}{ik_x - k_y}\right].
\end{equation}

Do dokaza indentiteta se dolazi supstitucijom Fourierovog derivacijskog teorema \ref{Fourier-derivative-theorem} u jednadžbu \ref{Morgan-T-definition}. 










\chapter{Sažimajuće očitavanje}
\section{Uzorkovanje signala}
\section{Mjerne matrice}\label{mjerne_matrice}
\section{Rekonstrukcija signala}

\chapter{Algoritam}


\section{Pregled sustava}
Zamišljeni optički sustav simuliran u ovom radu sastoji se od He-Ne\footnote{Helij-neon} lasera valne duljine $\lambda$ = 632.8 nm, jedne optičke rešetke, $m$ mjernih rešetki i senzora. Opisani sustav prikazan je na slici ??. 

Laserska zraka upada pod pravim kutem na optičku rešetku, stvarajući difrakcijske pruge na senzoru. Između rešetke i senzora postavlja se prozirni objekt, npr. tanka leća ili prizma, indeksa loma $n_{objekt} > n_{zrak}$. Razlika u sredstvu kroz koje svjetlost prolazi za posljedicu ima promjenu faze svjetlosti lasera. Korištenjem informacije o pomaku u fazi moguće je rekonstruirati debljinu odnosno oblik promatranog objekta. Rezolucija tj. broj piksela ovako dobivene slike je jednaka broju pukotina u optičkoj rešetci. Kako je osnovna ideja ovog rada pokazati da je sažimajućim očitavanjem iz prethodno opisanog postupka moguće dobiti sliku visoke rezolucije, između optičke rešetke i objekta se postavljaju mjerne rešetke. Mjerne rešetke su dizajnirane na poseban način, opisan u poglavlju \ref{mjerne_matrice}, kako bi rekonstrukcija postupkom sažimajućeg očitavanja bila moguća.


\chapter{Zaključak}
Zaključak.

\bibliography{literatura}
\bibliographystyle{unsrtnat}

\begin{sazetak}
Sažetak na hrvatskom jeziku.

\kljucnerijeci{Ključne riječi, odvojene zarezima.}
\end{sazetak}

% TODO: Navedite naslov na engleskom jeziku.
\engtitle{Title}
\begin{abstract}
Abstract.

\keywords{Keywords.}
\end{abstract}

\end{document}
